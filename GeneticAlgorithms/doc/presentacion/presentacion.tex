\documentclass{beamer}
\usepackage[utf8]{inputenc}
\usepackage[spanish]{babel}
\usepackage{graphicx}
\usepackage{verbatim}
\usepackage{moreverb}
\usepackage{amsmath}
\usepackage{amsfonts}
\usepackage{amssymb}
\usepackage{fancybox}
\usepackage{float}
\usepackage{fancyvrb}
\usepackage{color}
\usepackage{hyperref}
\usepackage{multirow}
\mode<presentation>
\usepackage{default}

\usetheme{Madrid}
\providecommand{\e}[1]{\ensuremath{\times 10^{#1}}}

\author{Grupo 5}
\title{Algoritmos Genéticos}
\subtitle{Trabajo Práctico Especial 4}
\institute[Sistemas de Inteligencia Artificial]{Sistemas de Inteligencia Artificial}

\begin{document}

\maketitle

\begin{frame}{Individiuo}
\begin{block}{Representación del individuo}
Vector de $\Re$  conformado por todos los pesos de la red, concatenando las filas de cada una de las matrices que representa las capas.
\end{block}

\begin{block}{Fidelidad}
\begin{itemize}
\item Completitud
\item Coherencia
\item Uniformidad
\item Sencillez
\item Localidad
\end{itemize}
\end{block}

\begin{block}{Población}
Se representa a la población como un \textit{ceil} de vectores individuos.
\end{block}
\end{frame}


\begin{frame}{Selección}
\begin{block}{Se implementaron los siguientes métodos}
\begin{itemize}
       \item Elitismo.
       \item Ruleta.
       \item Universal.
       \item Boltzmann.
       \item Elitismo + Ruleta
       \item Elitismo + Boltzmann
\end{itemize}
\end{block}

\begin{block}{Para Boltzman}
\begin{equation}
 T(t) = \frac{T_{\mbox{\emph{inicial}}}}{1 + \log t}
\end{equation}
\end{block}
\end{frame}

\begin{frame}{Fitness}
\begin{block}{Función de fitness}
Se busca maximizar la siguiente función de fitness:
\begin{equation}
       f(individuo) = \frac{1}{Err(individuo)}
\end{equation}

donde la función $Err(individuo)$ es el error cuadrático medio normalizado. Fitness aumenta a medida que el error disminuye.
\end{block}

\begin{block}{Función de fitness global}
\begin{equation}
      f_{\mbox{\emph{global}}}(P) = \sum_{1}^{n} f(P(i))
\end{equation}
\end{block}
\end{frame}

\begin{frame}{Operadores genéticos}
\begin{block}{Crossover}
    \begin{itemize}
     \item Clásico (un sólo punto)
     \item Múltiple
     \item Uniforme parametrizado
     \item Anular
    \end{itemize}
\end{block}

\begin{block}{Mutación}
    \begin{itemize}
      \item Clásica
      \item No uniforme
    \end{itemize}
\end{block}

\begin{block}{Backpropagation}
Es un operador específico para este problema. Se corren $k$ pasos \textit{feed-forward}, con una época de \textit{backpropagation}, con $k = 441$.
\end{block}
\end{frame}

\begin{frame}{Condición de corte}
\begin{block}{Condiciones de corte implementadas}
\begin{itemize}
       \item Cantidad máxima de generaciones.
       \item Entorno al óptimo.
       \item Estructura.
       \item Contenido.
\end{itemize}
\end{block}
\end{frame}

\begin{frame}{Pruebas}
\begin{block}{Parámetros}
\begin{itemize}
 \item Función de activación en las redes neuronales: \emph{tanh}.
 \item 441 puntos para el cálculo del \emph{fitness} y del \emph{backpropagation}.
\end{itemize}
\end{block}

\begin{block}{Notación}
\begin{itemize}
 \item $N$: Tamaño de la población.
 \item $p_m$: Probabilidad de mutación.
 \item $p_c$: Probabilidad de \emph{crossover}.
 \item $p_b$: Probabilidad de \emph{backpropagation}.
 \item $[m \; n \; o]$: red neuronal con una capa de entrada de $m$ neuronas una capa oculta de $n$ y una capa de salida de $o$.
 \item $Err_{f_{\mbox{\emph{max}}}}$: error cuadrático medio normalizado del individuo con mayor fitness de la última generación.
\end{itemize}
\end{block}
\end{frame}

\begin{frame}{Resultados: Prueba 1}
\begin{block}{Parámetros}
\begin{itemize}
\item Arquitectura: [2 4 2 1]
\item Selección y reemplazo: elitista
\item Crossover: clásico
\item Mutación: clásica
\item $p_b = 0.1 $
\item Condición de corte: cantidad máxima de generaciones de 200
\end{itemize}
\end{block}
\end{frame}

\begin{frame}{Resultados: Prueba 1}
\begin{block}{Tabla de resultados}
\begin{center}
\begin{tabular}{|c|c|c|c|c|c|}
 \hline
 $N$ & $p_m$ & $p_c$ & $G$ & $f_{\mbox{\emph{global}}}$ & $Err_{f_{\mbox{\emph{max}}}}$ \\
 \hline
 30 & 0.005 & 0.95 & 0.95 & 1394.94 & 7.17\e{-4} \\
 \hline
 30 & 0.01 & 0.6 & 0.75 & 1296.88 & 7.71\e{-4} \\
 \hline
 30 & 0.001 & 0.75 & 0.75 & 1435.42 & 6.97\e{-4} \\
 \hline
 30 & 0.001 & 0.95 & 0.75 & 1558.97 & 6.41\e{-4} \\
 \hline
 50 & 0.005 & 0.95 & 0.8 & 1914.69 & 5.22\e{-4} \\
 \hline
 50 & 0.01 & 0.6 & 0.85 & 1111.08 & 9\e{-4} \\
 \hline
 70 & 0.01 & 0.6 & 0.95 & 1728.77 & 5.78\e{-4} \\
 \hline
 70 & 0.001 & 0.95 & 0.95 & 1769.31 & 5.65\e{-4} \\
 \hline
 70 & 0.005 & 0.95 & 0.75 & 1398.98 & 7.15\e{-4} \\
 \hline
 130 & 0.005 & 0.75 & 0.75 & 2559.19 & 3.91\e{-4} \\
 \hline
 130 & 0.005 & 0.95 & 0.95 & 1949.53 & 5.13\e{-4} \\
 \hline
\end{tabular}
\end{center}
\end{block}
\end{frame}

\begin{frame}{Resultados: Prueba 2}
\begin{block}{Parámetros}
Se combinan distintos operadores de selección, mutación y \emph{crossover}.
\begin{itemize}
\item Arquitectura: [2 4 2 1]
\item Crossover: Clásico
\item Mutación: Clásica
 \item $N$: 130
 \item $p_m$: 0.005
 \item $p_c$: 0.75
 \item $p_b$: 0.1
 \item $G$: 0.75
\end{itemize}
\end{block}
\end{frame}

\begin{frame}{Resultados: Prueba 2}
\begin{block}{Tabla de resultados}

\begin{center}
\begin{tabular}{|c|c|c|c|c|c|}
 \hline
 \textbf{Selección} & \textbf{Reemplazo} & $p_c$ & $G$ & $f_{\mbox{\emph{global}}}$ & $Err_{f_{\mbox{\emph{max}}}}$ \\ 
 \hline
 Ruleta & Ruleta & 0.75 & 0.75 & 1490.09 & 6.57\e{-4} \\
 \hline
 E + R & Elite & 0.75 & 0.75 & 2493.55 & 4.01\e{-4} \\
 \hline
 Elite & Ruleta & 0.75 & 0.75 & 1641.21 & 6.09\e{-4} \\
 \hline
 Ruleta & Elite & 0.75 & 0.75 & 1688.37 & 5.92\e{-4} \\
 \hline
  Elite & E + B & 0.95 & 0.60 & 1005.01 & 1.986\e{-1} \\
 \hline
  Universal & E + B & 0.95 & 0.6 & 876,42 & 1.884\e{-1} \\
 \hline
   Universal & Universal & 0.95 & 0.65 & 1387,65 & 7.206\e{-4} \\
 \hline
   Elite & Universal & 0.95 & 0.75 & 137.864 & 7.2\e{-3} \\
 \hline
   Elite & Universal & 0.75 & 0.6 & 150.82 & 6.630\e{-3} \\
 \hline
     E + B & Elite & 0.95 & 0.6 & 1636.09 & 6.12\e{-4} \\
 \hline
\end{tabular}
\end{center}

\end{block}
\end{frame}

\begin{frame}{Conclusiones}
\begin{itemize}
 \item Se lograron resultados comparables a los obtenidos en el Trabajo Práctico 2, donde se había logrado un error de 4.8\e{-4}. Se puede destacar que
 los resultados se obtuvieron con arquitectura mucho más pequeña.
 
 \item Se obtuvieron buenos resultados en todas las corridas, a diferencia del entrenamiento con \emph{backpropagation} que muchas veces se atascaba en un mínimo local.
\end{itemize}
\end{frame}

\end{document}
